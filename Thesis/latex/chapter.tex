{\let\clearpage\relax \chapter{Objective}}
This thesis aims to critically evaluate the effectiveness of contemporary multilingual topic modeling techniques in extracting meaningful insights from extensive multilingual Twitter datasets, specifically focusing on the discourse surrounding the Russia-Ukraine War. It seeks to determine whether these methods are adept at navigating the complexities and nuances inherent in large-scale, multilingual social media data, thereby offering a robust tool for understanding the dynamics of digital communication in the context of international conflicts.

To achieve this objective, the thesis will first process and analyze a substantial multilingual Twitter dataset obtained from Kaggle, focusing on content related to the Russian-Ukraine conflict. Subsequently, it will provide a comprehensive overview of the current developments in multilingual topic modeling, highlighting the advancements and challenges in this field. Finally, the study will apply suitable multilingual topic modeling techniques to the dataset, aiming to effectively discern and interpret the patterns and themes that emerge from this complex digital discourse.
